\documentclass[12pt]{article}
 
\usepackage[margin=1in]{geometry} 
\usepackage{amsmath,amsthm,amssymb}
\usepackage{graphicx} 
\usepackage{listings}
\usepackage{color}

\definecolor{dkgreen}{rgb}{0,0.6,0}
\definecolor{gray}{rgb}{0.5,0.5,0.5}
\definecolor{mauve}{rgb}{0.58,0,0.82}

\lstset{frame=tb,
  language=Python,
  aboveskip=3mm,
  belowskip=3mm,
  showstringspaces=false,
  columns=flexible,
  basicstyle={\small\ttfamily},
  numbers=none,
  numberstyle=\tiny\color{gray},
  keywordstyle=\color{blue},
  commentstyle=\color{dkgreen},
  stringstyle=\color{mauve},
  breaklines=true,
  breakatwhitespace=true,
  tabsize=2
}

\begin{document}
 
\title{Intro to Cryptography} 
\author{Mark Anderson\\ 
Problem 6} 
 
\maketitle
\begin{enumerate}
  \item For implementing the Rabin cryptosystem, I was able to successfully encrypt and decrypt the example given in the book.  However I ran into a similar problem as in question 7 of my method for encryption/decryption relied on the calculation of very large primes, specifically $T^{p+1 // 4} mod p$ and even decrypting one letter with the large primes given in the prompt was taking too long.  With the very small values of 127, and 131 it worked just fine, I was able to encrypt '4410' and retrieve the plaintexts: 4410, 5851, 15078, 16519.  I believe I am missing a step somewhere that drastically reduces the size of that exponent, maybe a modular reduction or even a completely different number.  The encryption is working fine, and even with the incredibly large primes is insanely fast, as its just quick multiplication and modulus.
\end{enumerate}



\end{document}
