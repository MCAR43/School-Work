\documentclass[12pt]{article}
 
\usepackage[margin=1in]{geometry} 
\usepackage{amsmath,amsthm,amssymb}
\usepackage{graphicx} 
\begin{document}
 
\title{Artficial Intelligence} 
\author{Mark Anderson\\ 
Homework 4} 
 
\maketitle
On Page 90 we mentioned iterative lengthening search, an iterative analog of uniform cost search.  The idea is to use increasing limits on path cost.  If a node is generated whose path cost exceeds the current limit, it is immediately discarded.  For each new iteration, the limit is set to the lowest path cost of any node discarded in the previous iteration.
\begin{enumerate}
  \item Show that this algorithm is optimal for general path costs \par
    \begin{itemize}
      \item Once the Goal state is reached, the only available paths will be those with the lowest cost, if a lower path cost existed it would've been generated in the previous iteration. 
    \end{itemize}
  \item Consider a uniform tree with branching factor b, solution depth d, and unit step costs.  How many iterations will iterative lengthening require?
    \begin{itemize}
      \item Iterative Lengthening will require the same number of iterations as iterative deepening in this case, which is D iterations where the number of nodes is \[d^b\] this is because with unit step costs and a uniform branching factor the algorithm functions similarly to iterative deepening.
    \end{itemize}
  \item Now consider step costs drawn from the contiunous range [e,1] where 0 $<$ e $<$ 1.  How many iterations are required in the worst case?
    \begin{itemize}
      \item The worst case scenario will be the depth divided by the maximum value of e in 0 < e < 1, because if the goal node is at the maximum depth, and at a step cost of e, it will be the last node to be reached. 
    \end{itemize}

\end{enumerate}
\end{document}
